\documentclass[titlepage,11pt]{article}
\usepackage[margin=1in]{geometry}
\usepackage{enumitem}
\usepackage{titlesec}
\usepackage{textcomp}
\usepackage{amsmath}

\title{Final Project Proposal}
\author{Robinson Merillat, Matt Clayton, Michael Niemeyer}
\date{\today}

\begin{document}
\maketitle

The stage is set, the people are in place, it's the morning of the grand opening of the 
park! Ra greets you at the front gate. "Well done!" he says. "You have truly proved yourself
as an exemplary architect time and again. In fact, you may even be worthy of the title of a 
god! I have but one last task for you, if you should complete it, you shall have a place 
with us." With that, he hands you a scroll and walks off to discuss matters with other
architects that have been assisting with the park's development. You open the scroll,
it reads....\\

\noindent\fbox{%
    \parbox{\textwidth}{%
        \begin{center}
            Welcome to the real light show!\\
        \end{center}
        
        I am quite pleased with the light show that you created that will be performed daily at 
        the park. However, for the grand opening this evening, I will be inviting a live band to 
        kick things off and we need a light show that is truly one of a kind! I have tasked you to 
        work with two other architects to complete this task, best of luck and godspeed!
    }%
}

\section{Description}

The goal of this project is to create a light show "music video" with the use of a series of particle
systems. Various particle systems will emit as a light source while others do not (fountain vs. firework).
Systems that emit light will affect other objects within the scene. The setting of the light show will be 
at Auru Park! Complete with buildings (lab 12), a pyramid, and the very roller coaster that we designed! 
The light show can play on loop or be stopped. At any time the user can use various keys on the 
keyboard to make different particle systems fire off. If the user has had enough of watching the light show 
from afar, they can go first person as a textured model (possibly hellknight) to run around the park.
Careful though, if the hellknight stops moving, he may start to dance (intended easter egg #1).

Various textures and materials will be used for buildings/pyramid and the ground around the park.
The entirety of the park will be held within a skysphere (skybox as a fallback)

Many different shaders will be used to generate different effects for the different types of 
particle systems that are planned to be developed. These include but are not limited to: laser lights,
water fountain, fireworks (the big boomers), and sparks.

As a final goal, we would like to create a video recording of the light show with music overlayed as
a park advertisement.

\section{Forseeable Challenges}
Some forseeable challenges include:
\begin{itemize}
    \item opengl 3.3 bezier curve (camera movement along curve)
    \item modifying skeletal animation for hellknight or similar model (walking/dancing)
    \item multiple lights acting on multiple objects in the  scene
    \item skysphere vs skybox
    \item lining up light show with song
\end{itemize}

\end{document}
